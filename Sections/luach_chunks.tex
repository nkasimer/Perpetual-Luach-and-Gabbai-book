\newcommand{\erevRH}{On Erev Rosh HaShana the Seli\d{h}ot for the day are recited, ideally before sunrise. Ta\d{h}anun is omitted in sha\d{h}arit and min\d{h}a. LeDavid is recited at Sha\d{h}arit, but the Shofar is not sounded. Some fast on this day. Many say Hatarat Nedarim, and immerse in a mikveh.}

	\newcommand{\roshhashanashabbat}{
	\textbf{Rosh Hashana I \textendash\space 1 Tishrei}
	%date.7.1
	
	Kabbalat Shabbat is abbreviated, beginning with \hebword{מזמור שיר ליום השבת}.  The chapter \hebword{במה מדליקין}\space is omitted. \textit{Le'eila le'eila} is included in Kaddish beginning with the Kaddish following abbreviated Kabbalat Shabbat. Arvit is chanted in the appropriate tune for Rosh Hashana.   Both \textit{Veshameru} and \textit{tik`u ba\d{h}odesh shofar} are recited before \halfkad. The Amidah is recited with \hebword{המלך הקדוש} and \hebword{עושה השלום}.  These are continued through Yom Kippur. \hebword{ויכולו}\space and the Berakha E\d{h}ad Me'ein Shalosh is chanted in the Shabbat tune, with \hebword{המלך הקדוש}.
	
	Kiddush is the text for Rosh Hashana, with additions for Shabbat, followed by \hebword{שהחיינו}.  LeDavid is recited at the end of Arvit.
	
	Sha\d{h}arit is recited according to the text in the ma\d{h}zor, with piyyutim according to local communal custom. \hebword{אל אדון} is recited in sha\d{h}arit. The Shir Shel Yom is for Shabbat, along with LeDavid.  Many communities recite Anim Zemirot (either because it is Rosh Hashana or because it is Shabbat).  Two Sifrei Torah are taken from the Ark.  The Torah reading is the story of Isaac's birth and Ishma`el being expelled from Abraham's household, divided into seven aliyot, from the first Sefer Torah, and the Maftir is from the holiday listing in Pin\d{h}as.
	
	\leinRoshHashanaIonShabbat
	
	\halfkad is recited, and the first Sefer Torah is lifted and wrapped. Maftir is read from the second Sefer Torah. \haftRoshHashanaI The berakha after the haftara uses the text for Rosh Hashana with additions for Shabbat.
	
	Shofar is not blown.  Musaf includes the additions for Shabbat.  \hebword{וביום השבת}\space and \hebword{ישמחו}\space should be chanted in the tune for Shabbat.
	
	Afternoon kiddush includes the verses for both Shabbat and Rosh Hashana.
	
	Min\d{h}a begins with Ashrei, Uva Letzion, \halfkad, and the Torah reading for Ha`azinu. 
	
	\leinHaazinuweekday
	
	After the Torah is returned, \halfkad is said, and the Amida for Rosh Hashana is recited. \hebword{צדקתך צדק}\space is omitted.
	
	\textbf{Rosh Hashana II \textendash\space 2 Tishrei}
	%date.7.2
	
	Arvit for Rosh Hashana is recited, beginning with Berekhu.  The havdala paragraph is inserted in the Amida.  \yakenhaz . Since the two days of Rosh Hashana are considered single ``long day'', the custom is that a new fruit is eaten or a new garment is worn so that the berakha of shehe\d{h}iyanu applies to it as well as the holiday. LeDavid is recited after Aleinu.
	
	Sha\d{h}arit is recited according to the text in the ma\d{h}zor, with piyyutim according to local communal custom. \hebword{המאיר} is recited in sha\d{h}arit. The Shir Shel Yom is for Sunday, along with LeDavid.  Some communities recite Anim Zemirot.  Two Sifrei Torah are taken from the Ark.  The Torah reading is the story of the binding of Isaac, which immediately follows the reading from the first day.
	
	\leinRoshHashanaII
	
	\halfkad is recited, and the first Sefer Torah is lifted and wrapped. Maftir is read from the second Sefer Torah. \haftRoshHashanaII The berakha after the haftara uses the text for Rosh Hashana.
	
	Shofar is blown, with \textit{shehe\d{h}iyanu}. Three sets each of each sequence of blasts are blown.  Musaf is recited according to the text in the ma\d{h}zor with piyyutim according to local custom. Some communities blow the shofar during the silent Amidah.  The shofar is blown during the repetition.  Most communities blow a total of 100 blasts of the shofar, which requires additional blasts after the amida.  Most do this before \hebword{תתקבל} in the Kaddish Shalem following the Amida.
	
	Afternoon kiddush for Rosh Hashana.
	
	Min\d{h}a begins with Ashrei and Uva Letzion. \halfkad and the Amida for Rosh Hashana, Avinu Malkeinu, \fullkad , etc.
	
	\textbf{3 Tishrei}
	%date.7.3
	
	The holiday is concluded with Arvit for weekdays. Vihi No`am is not recited. Havdala is on wine only, without introductory verses, flame or spices.
}

\newcommand{\roshhashana}[3]{
	\textbf{Rosh Hashana I \textendash\space 1 Tishrei}
%date.7.1

Arvit is chanted in the appropriate tune for Rosh Hashana.  \textit{Le'eila le'eila} is included in Kaddish. \textit{Tik`u ba\d{h}odesh shofar} are recited before \halfkad. The Amidah is recited with \hebword{המלך הקדוש} and \hebword{עושה השלום}.  These are continued through Yom Kippur.

Kiddush is the text for Rosh Hashana, followed by \hebword{שהחיינו}.  LeDavid is recited at the end of Arvit.

Sha\d{h}arit is recited according to the text in the ma\d{h}zor, with piyyutim according to local communal custom. \hebword{המאיר} is recited in sha\d{h}arit. The Shir Shel Yom is for #1, along with LeDavid.  Some communities recite Anim Zemirot.  Two Sifrei Torah are taken from the Ark.  The Torah reading is the story of Isaac's birth and Ishma`el being expelled from Abraham's household, divided into five aliyot, from the first Sefer Torah, and the Maftir is from the holiday listing in Pin\d{h}as.

\leinRoshHashanaI

\halfkad is recited, and the first Sefer Torah is lifted and wrapped. Maftir is read from the second Sefer Torah. \haftRoshHashanaI The berakha after the haftara uses the text for Rosh Hashana.

Shofar is blown, with \textit{shehe\d{h}iyanu}. Three sets each of each sequence of blasts are blown.  Musaf is recited according to the text in the ma\d{h}zor with piyyutim according to local custom. Some communities blow the shofar during the silent Amidah.  The shofar is blown during the repetition.  Most communities blow a total of 100 blasts of the shofar, which requires additional blasts after the amida.  Most do this before \hebword{תתקבל} in the Kaddish Shalem following the Amida.

Afternoon kiddush for Rosh Hashana.

Min\d{h}a begins with Ashrei and Uva Letzion. \halfkad and the Amida for Rosh Hashana, Avinu Malkeinu, \fullkad , etc.

\textbf{Rosh Hashana II \textendash\space 2 Tishrei}
%date.7.2

Arvit for Rosh Hashana is recited, beginning with Berekhu.  Since the two days of Rosh Hashana are considered single ``long day'', the custom is that a new fruit is eaten or a new garment is worn so that the berakha of shehe\d{h}iyanu in kiddush applies to it as well as the holiday. LeDavid is recited after Aleinu.

Sha\d{h}arit is recited according to the text in the ma\d{h}zor, with piyyutim according to local communal custom. \hebword{המאיר} is recited in sha\d{h}arit. The Shir Shel Yom is for #2, along with LeDavid.  Some communities recite Anim Zemirot.  Two Sifrei Torah are taken from the Ark.  The Torah reading is the story of the binding of Isaac, which immediately follows the reading from the first day.

\leinRoshHashanaII

\halfkad is recited, and the first Sefer Torah is lifted and wrapped. Maftir is read from the second Sefer Torah. \haftRoshHashanaII The berakha after the haftara uses the text for Rosh Hashana.

Shofar is blown, with \textit{shehe\d{h}iyanu}. Three sets each of each sequence of blasts are blown.  Musaf is recited according to the text in the ma\d{h}zor with piyyutim according to local custom. Some communities blow the shofar during the silent Amidah.  The shofar is blown during the repetition.  Most communities blow a total of 100 blasts of the shofar, which requires additional blasts after the amida.  Most do this before \hebword{תתקבל} in the Kaddish Shalem following the Amida.

Afternoon kiddush for Rosh Hashana.

Min\d{h}a begins with Ashrei and Uva Letzion. \halfkad and the Amida for Rosh Hashana, Avinu Malkeinu, \fullkad , etc.

\textbf{3 Tishrei}
%date.7.3
The holiday is concluded with Arvit for weekdays. Vihi No`am is not recited. Havdala is on wine only, without introductory verses, flame or spices.}

\newcommand{\roshhashanaThursday}{
	\textbf{Rosh Hashana I \textendash\space 1 Tishrei}
%date.7.1

Arvit is chanted in the appropriate tune for Rosh Hashana.  \textit{Le'eila le'eila} is included in Kaddish. \textit{Tik`u ba\d{h}odesh shofar} are recited before \halfkad. The Amidah is recited with \hebword{המלך הקדוש} and \hebword{עושה השלום}.  These are continued through Yom Kippur.

Kiddush is the text for Rosh Hashana, followed by \hebword{שהחיינו}.  LeDavid is recited at the end of Arvit.

Sha\d{h}arit is recited according to the text in the ma\d{h}zor, with piyyutim according to local communal custom. \hebword{המאיר} is recited in sha\d{h}arit. The Shir Shel Yom is for Thursday, along with LeDavid.  Some communities recite Anim Zemirot.  Two Sifrei Torah are taken from the Ark.  The Torah reading is the story of Isaac's birth and Ishma`el being expelled from Abraham's household, divided into five aliyot, from the first Sefer Torah, and the Maftir is from the holiday listing in Pin\d{h}as.

\leinRoshHashanaI

\halfkad is recited, and the first Sefer Torah is lifted and wrapped. Maftir is read from the second Sefer Torah. \haftRoshHashanaI The berakha after the haftara uses the text for Rosh Hashana.

Shofar is blown, with \textit{shehe\d{h}iyanu}. Three sets each of each sequence of blasts are blown.  Musaf is recited according to the text in the ma\d{h}zor with piyyutim according to local custom. Some communities blow the shofar during the silent Amidah.  The shofar is blown during the repetition.  Most communities blow a total of 100 blasts of the shofar, which requires additional blasts after the amida.  Most do this before \hebword{תתקבל} in the Kaddish Shalem following the Amida.

Afternoon kiddush for Rosh Hashana.

Min\d{h}a begins with Ashrei and Uva Letzion. \halfkad and the Amida for Rosh Hashana, Avinu Malkeinu, \fullkad , etc.

\textbf{Rosh Hashana II \textendash\space 2 Tishrei}
%date.7.2

Arvit for Rosh Hashana is recited, beginning with Berekhu.  Since the two days of Rosh Hashana are considered single ``long day'', the custom is that a new fruit is eaten or a new garment is worn so that the berakha of shehe\d{h}iyanu in kiddush applies to it as well as the holiday. LeDavid is recited after Aleinu.

Sha\d{h}arit is recited according to the text in the ma\d{h}zor, with piyyutim according to local communal custom. \hebword{המאיר} is recited in sha\d{h}arit. The Shir Shel Yom is for Friday, along with LeDavid.  Some communities recite Anim Zemirot.  Two Sifrei Torah are taken from the Ark.  The Torah reading is the story of the binding of Isaac, which immediately follows the reading from the first day.

\leinRoshHashanaII

\halfkad is recited, and the first Sefer Torah is lifted and wrapped. Maftir is read from the second Sefer Torah. \haftRoshHashanaII The berakha after the haftara uses the text for Rosh Hashana.

Shofar is blown, with \textit{shehe\d{h}iyanu}. Three sets each of each sequence of blasts are blown.  Musaf is recited according to the text in the ma\d{h}zor with piyyutim according to local custom. Some communities blow the shofar during the silent Amidah.  The shofar is blown during the repetition.  Most communities blow a total of 100 blasts of the shofar, which requires additional blasts after the amida.  Most do this before \hebword{תתקבל} in the Kaddish Shalem following the Amida.

Afternoon kiddush for Rosh Hashana.

Min\d{h}a begins with Ashrei and Uva Letzion. \halfkad and the Amida for Rosh Hashana. Avinu Malkeinu is omitted, \fullkad , etc.
}

\newcommand{\tzomGedalia}[3]{
\textbf{Tzom Gedalia \textendash\space #3 Tishrei}

The Third of Tishrei is a minor fast, commemoratting the assassination of Gedalia, the governor of the Babylonian province of Judea.  His assassination ended the last Jewish governance that had remained after the destruction of the First Temple. It also is part of the fasting of the Yamim Noraim season, which formerly included days of fasting. #2

Seli\d{h}ot are recited (before sha\d{h}arit, as during the 10 Days of Repentence). Additions to the Amidah for both minor fasts and the 10 Days of Repentence are included. Avinu Malkeinu is read, using the text for the 10 Days of Repentence.

\leinminorfast

Psalm of the day for #1 and Ledavid.

Min\d{h}a is said with the same Torah reading as sha\d{h}arit, except that the third aliya is called as Maftir. \haftTzomMincha The Amidah is recited, with additions for the Ten Days of Repentence and minor fasts. Avinu Malkeinu is recited, with the text for the 10 Days of Repentence.
}

\newcommand{\shabbatshuvafirsthalf}{Kabbalat Shabbat and Arvit as usual, with the additions to the Amidah for the 10 Days of Repentence, and \textit{hamelekh hakadosh} in the Berakha Achat Me`ein Sheva, concluding with LeDavid.

Shabbat Sha\d{h}arit, with additions for the 10 Days of Repentence in the Amidah.  Most communities open the ark and recite \hebword{שיר המעלות ממעמקים} after \textit{Yishtaba\d{h}}.  LeDavid is recited.
}

\newcommand{\vayelechShuva}[1]{
\textbf{Shabbat Shuva {Vayelech} \textendash\space #1 Tishrei}

\shabbatshuvafirsthalf

\leinHaazinu

Customs vary for the Haftara.  All communities begin with Hosea 14:2–10.  Some add Joel 2:11–27 and Micah 7:18–20 (note that Joel should be before Micah to avoid going backwards within the 12 Prophets), others add only the verses from Micah when Vayelech is read on Shabbat Shuva. Some always add the verses from Joel or Micah.

Shabbat Min\d{h}a begins as usual, reading from Haazinu. The Amida includes additions for the 10 Days of Repentence. \hebword{צדקתך} is said.

Arvit for the conclusion of Shabbat as usual.  \textit{Vihi no`am} is omitted.
}
\newcommand{\haazinuShuva}[1]{
\textbf{Shabbat Shuva {Haazinu} \textendash\space #1 Tishrei}

\shabbatshuvafirsthalf

\leinHaazinu

Customs vary for the Haftara.  All communities begin with Hosea 14:2–10.  Some add Joel 2:11–27 and Micah 7:18–20 (note that Joel should be before Micah to avoid going backwards within the 12 Prophets), others add only the verses from Joel when Haazinu is read on Shabbat Shuva. Some always add the verses from Joel or Micah.

Shabbat Min\d{h}a begins as usual, reading from VeZot HaBerakha.

The Amida includes additions for the 10 Days of Repentence. \hebword{צדקתך} is said.

Arvit for the conclusion of Shabbat as usual.  \textit{Vihi no`am} is said.
}
\newcommand{\haazinuShuvaMotzeiRH}{Kabbalat Shabbat is abbreviated, since there is a festival on Friday. Arvit has additions to the Amidah for the 10 Days of Repentence, and \textit{hamelekh hakadosh} in the Berakha Achat Me`ein Sheva, concluding with LeDavid.
	
	Shabbat Sha\d{h}arit, with additions for the 10 Days of Repentence in the Amidah.  Most communities open the ark and recite \hebword{שיר המעלות ממעמקים} after \textit{Yishtaba\d{h}}.  LeDavid is recited. 
	
	\leinHaazinu
	
	Customs vary for the Haftara.  All communities begin with Hosea 14:2–10.  Some add Joel 2:11–27 and Micah 7:18–20 (note that Joel should be before Micah to avoid going backwards within the 12 Prophets), others add only the verses from Joel when Haazinu is read on Shabbat Shuva. Some always add the verses from Joel or Micah.
	
	Shabbat Min\d{h}a begins as usual, reading from VeZot HaBerakha. The Amida includes additions for the 10 Days of Repentence. \hebword{צדקתך} is said.

Arvit for the conclusion of Shabbat as usual.  \textit{Vihi no`am} is said.}

\newcommand{\normalshabbosarvit}{Kabbalat Shabbat and Arvit as usual.\space}
\newcommand{\normalshabbatshacharit}{Pesukei DeZimra and Sha\d{h}arit as usual.\space}
\newcommand{\normalshabbatmusaf}{Shabbat Musaf as usual.}

\newcommand{\mevarekhim}[4]{The new month of #1 is blessed. See page \pageref{molads} for the molad. Rosh \d{H}odesh is #2.\\
\hebword{ראש חודש #3 יהיה #4}
}

\newcommand{\shabbatbereshit}[5]{%date, haftara, days of RCh eng, days of RCh heb, omit tzidkaskha
	\textbf{Shabbat Bereshit \textendash\space #1 Tishrei}

\normalshabbosarvit \normalshabbatshacharit
\leinBereshit

Note that different \d{h}umashim have different aliya breakdowns for this parasha. Make sure to check with leiners that they learned the correct portion. #2

\mevarekhim{Mar\d{h}eshvan (not ``Cheshvan'')}{#3}{מרחשון}{#4}. Av HaRa\d{h}amim is omitted. \normalshabbatmusaf

Min\d{h}a for Shabbat. #5
}
\newcommand{\shabbereshitmacharchodesh}{\shabbatbereshit{29}{\haftMacharChodesh Some read the Haftara for Bereishit even when it falls on Ma\d{h}ar \d{H}odesh (Isaiah 42:5\textendash 43:10)}{Sunday and Monday}{ביום ראשון וביום שני}{\textit{Tzidkatekha} is omitted for Erev Rosh \d{H}odesh.}
}

\newcommand{\tishreiShab}{
	\roshhashanashabbat
	
	\tzomGedalia{Monday}{}{3}

	
	Following Tzom Gedalia, liturgical additions continue for the 10 Days of Repentence.
	
	\haazinuShuva{8}
	
	%tishrei holidays here
	\shabbereshitmacharchodesh
}

\newcommand{\tishreiMon}{
\roshhashana{Monday}{Tuesday}{The holiday is concluded with Arvit for weekdays. Vihi No`am is not recited. Havdala is on wine only, without introductory verses, flame or spices.}

\tzomGedalia{Wednesday}{}{3}

\vayelechShuva{6}

%tishrei holidays here
}

\newcommand{\tishreiTues}{
\roshhashana{Tuesday}{Wednesday}

\tzomGedalia{Thursday}{}{3}

\vayelechShuva{5}
}

\newcommand{\tishreiThurs}{
\roshhashanaThursday

\haazinuShuvaMotzeiRH

\tzomGedalia{Sunday}{Because the 3rd of Tishrei falls on Shabbat, the fast is pushed off to Sunday.}{4}
}