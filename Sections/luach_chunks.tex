\newcommand{\halfkad}{\d{H}atzi Kaddish\space}

\newcommand{\yakenhaz}{Kiddush is YaKeNHaZ\textemdash Yayin (hagafen), Kiddush (\hebword{אשר בחר בנו וכו׳}), Ner (\hebword{בורא מאורי האש}), Havdala (\hebword{המבדיל בין קודש לקודש}), Zeman (\hebword{שהחיינו}).}

\newcommand{\roshhashanashabbatleining}{\begin{tabular}{c | c | c}
		Kohen &Genesis 21:1-4 & 4 pesukim\\\hline
		Levi &Genesis 21:5-8 & 4 pesukim\\\hline
		Shelishi & Genesis 21:9-12 & 4 pesukim\\\hline
		Revi`i &Genesis 21:13-17 & 5 pesukim\\\hline
		\d{H}amishi &Genesis 21:18-21 &4 pesukim\\\hline
		Shishi &Genesis 21:22-27 &6 pesukim\\\hline
		Shevi`i &Genesis 21:28-34 & 7 pesukim\\
\end{tabular}}
\newcommand{\roshhashanatwoleining}{\begin{tabular}{c | c | c}
		Kohen &Genesis 22:1-3 & 3 pesukim\\\hline
		Levi &Genesis 22:4-8 & 5 pesukim\\\hline
		Shelishi & Genesis 22:9-14 & 6 pesukim\\\hline
		Revi`i &Genesis 22:15-19 & 5 pesukim\\\hline
		\d{H}amishi &Genesis 22:20-24 &5 pesukim\\
	\end{tabular}
}

\newcommand{\roshhashanashabbat}{
On Erev Rosh HaShana the Seli\d{h}ot for the day are recited, ideally before sunrise. Ta\d{h}anun is omitted in sha\d{h}arit and min\d{h}a. LeDavid is recited at Sha\d{h}arit, but the Shofar is not sounded. Some fast on this day.

\textbf{1 Tishrei}

Kabbalat Shabbat is abbreviated, beginning with \hebword{מזמור שיר ליום השבת}.  The chapter \hebword{במה מדליקין}\space is omitted. Arvit is chanted in the appropriate tune for Rosh Hashana.  \textit{Le'eila le'eila} is included in Kaddish. Both \textit{Veshameru} and \textit{tik`u ba\d{h}odesh shofar} are recited before \halfkad. The Amidah is recited with \hebword{המלך הקדוש} and \hebword{עושה השלום}.  These are continued through Yom Kippur. \hebword{ויכולו}\space and the Berakha E\d{h}ad Me'ein Shalosh is chanted in the Shabbat tune, with \hebword{המלך הקדוש}.

Kiddush is the text for Rosh Hashana, with additions for Shabbat, followed by \hebword{שהחיינו}.  LeDavid is recited at the end of Arvit.

Sha\d{h}arit is recited according to the text in the ma\d{h}zor, with piyyutim according to local communal custom. \hebword{אל אדון} is recited in sha\d{h}arit. The Shir Shel Yom is for Shabbat, along with LeDavid.  Many communities recite Anim Zemirot (either because it is Rosh Hashana or because it is Shabbat).  Two Sifrei Torah are taken from the Ark.  The Torah reading is the story of Isaac's birth and Ishma`el being expelled from Abraham's household, divided into seven aliyot.

\roshhashanashabbatleining

\halfkad is recited, and the first Sefer Torah is lifted and wrapped. Maftir is read from the second Sefer Torah, Numbers 29:1-6.  The Haftarah is the story of \d{H}anna, I Samuel 1:1-2:10. The berakha after the haftara uses the text for Rosh Hashana with additions for Shabbat.

Shofar is not blown.  Musaf includes the additions for Shabbat.  \hebword{וביום השבת}\space and \hebword{ישמחו}\space should be chanted in the tune for Shabbat.

Afternoon kiddush includes the verses for both Shabbat and Rosh Hashana.

Min\d{h}a begins with Ashrei, Uva Letzion, \halfkad, and the Torah reading for Ha`azinu. After the Torah is returned, \halfkad is said, and the Amida for Rosh Hashana is recited. \hebword{צדקתך צדק}\space is omitted.

\textbf{2 Tishrei}

Arvit for Rosh Hashana is recited, beginning with Berekhu.  The havdala paragraph is inserted in the Amida.  \yakenhaz . Since the two days of Rosh Hashana are considered single ``long day'', the custom is that a new fruit is eaten or a new garment is worn so that the berakha of shehe\d{h}iyanu applies to it as well as the holiday. LeDavid is recited after Aleinu.

Sha\d{h}arit is recited according to the text in the ma\d{h}zor, with piyyutim according to local communal custom. \hebword{המאיר} is recited in sha\d{h}arit. The Shir Shel Yom is for Sunday, along with LeDavid.  Some communities recite Anim Zemirot.  Two Sifrei Torah are taken from the Ark.  The Torah reading is the story of the binding of Isaac, which immediately follows the reading from the first day.

\roshhashanatwoleining

\halfkad is recited, and the first Sefer Torah is lifted and wrapped. Maftir is read from the second Sefer Torah, Numbers 29:1-6.  The Haftarah is Jeremiah 31:1-19. The berakha after the haftara uses the text for Rosh Hashana.

Shofar is blown, with \hebword{shehe\d{h}iyanu}. Three sets each of each sequence of blasts are blown.  Musaf is recited according to the text in the ma\d{h}zor with piyyutim according to local custom. Some communities blow the shofar during the silent Amidah.  The shofar is blown during the repetition.  Most communities blow a total of 100 blasts of the shofar, which requires additional blasts after the amida.  Most do this before \hebword{תתקבל} in the Kaddish Shalem following the Amida.

Afternoon kiddush for Rosh Hashana.

Min\d{h}a begins with Ashrei and Uva Letzion. \halfkad and the Amida for Rosh Hashana, Avinu Malkeinu, \fullkad

}

\newcommand{\tishreiShab}{
\roshhashanashabbat
}