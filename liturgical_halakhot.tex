\documentclass[11pt]{article}
\usepackage[paperheight=6.875in,paperwidth=4.5in,top=.5in,bottom=.5in, inner=.875in, outer=.75in, marginparsep=.1in, headsep=16pt]{geometry}

\usepackage{polyglossia}
\usepackage{hyperref}
\usepackage{multicol}
\usepackage{etoolbox}
\newtoggle{breakintorah}
\togglefalse{breakintorah}
\setdefaultlanguage{english}
\setotherlanguage{hebrew}
\usepackage{fontspec}
\usepackage[backend=bibtex,style=authortitle-ibid]{biblatex}
\bibliography{lit_halakhot_sources}
\usepackage{hyperref}

\setmainfont{EB Garamond}

\newfontfamily\hebrewfont{Shlomo}

\newcommand{\hebword}[1]{‎\begin{hebrew}\beginR #1 \endR\end{hebrew}}

\newcommand{\heth}{\d{h}}
\newcommand{\bigheth}{\d{H}}
\newcommand{\ayin}{Ȝ}
\newcommand{\smallayin}{ȝ}

\newcommand{\amidah}{Amidah\space}
\newcommand{\amidahnospace}{Amidah}
\newcommand{\aliya}{aliya}
\newcommand{\aliyot}{aliyot}
\newcommand{\SA}{Shul\d{h}an Arukh}
\newcommand{\arvit}{Arvit}

\usepackage[parfill]{parskip}
\begin{document}
	
\title{A Brief Guide to the Laws of Prayer}

\author{Compiled by Nathan Kasimer}

\date{}

\maketitle


\section{Introduction}

This guide is intended as a quick-reference for liturgical questions that come up frequently and require an immediate answer, and are more in-depth than siddur instructions typically have. It therefore omits discussion of the obligation of prayer, times for the different liturgies, attire for prayer, and other more regularly-occurring matters of ritual law. It also omits significant amounts of halakhic discussion and relevant details for the sake of conciseness.

Given that this guide is a summary of summaries, it should not be used as an authoritative halakhic guide. When possible a more comprehensive text should be consulted.  This text is designed for when doing so would cause undo interruption or a burden on the congregation.

The guide is primarily based on the English translation of Peninei Halakha.  When it cites major codes, the citations are referenced here. This has been supplemented by references to the Kitzur \SA\space and other sources as noted. Citations to the \SA\space are abbreviated to the section  (``OC" or ``YD") and Siman.

\section{Situations where Prayer is Forbidden}

It is forbidden to pray near uncovered excrement, smelly garbage, or other foul smells.  A person must be four cubits (approximately 2 meters or 6$\frac{1}{2}$ feet) from the source of the smell or where they cannot smell it, whichever is further, even if they themselves cannot smell it \parencite*[3:9 citing OC 79]{PH}.%(PH Prayer 3:9 citing OC 79).

The above does not apply to feces of babies who do not eat much solid food, since it does not smell as bad as that of adults or animals \parencite*[4:3 citing OC 81]{PH}. Diapers, catheters, and the like are considered ``covered" \parencite*[4:4]{PH}.

One may not hold an item they are afraid will fall during prayer, unless they would be afraid the item would be stolen and that would interfere with proper attention to praying \parencite[5:7 citing OC 96]{PH}.

One may not pray when they need to use the toilet \parencite*[5:8 citing Berakhot 15a]{PH}.

A person who is tipsy should not pray, but if they do so their obligation is fulfilled.  A person who is inebriated may not pray \parencite*[5:11 citing OC 99]{PH}.

\section{A Diminished Minyan}

If a portion of the minyan leaves during any prayer requiring a minyan, the prayer may be completed as long as the majority of the minyan remains  \parencite*[2:10 citing OC 55]{PH}.

The repetition of the \amidah is considered the same prayer, but Birkat Kohanim and the Torah service are not. In Ashkenazi practice, kaddishes after the \amidah up to and including the following Full Kaddish are considered the same prayer. The blessings of Shema and the \amidah that follows are considered different prayers \parencite*{PH}.

If a minyan was absent for Pesukei DeZimra and Yishtaba\heth\space was recited, and then a minyan was formed, a few verses should be read before Half Kaddish is recited \parencite*[15:1]{Kitzur}

\section{Interruptions in Liturgy}

\subsection{During Pesukei DeZimra}

Pesukei DeZimra is the portion of the service between Barukh She'amar and Yishtaba\heth . Needless interruptions are forbidden during that period, with the following exceptions (based on \cite*[``Table of Permitted Responses"]{Koren} and \cite*[16:5]{PH}):
\begin{itemize}
	\item Answering ``amen" to a blessing 
	\item The congregational response to kedusha and barekhu
	\item Reciting modim derabbanan
	\item Answering \hebword{אמן יהי שמה רבא וכו׳} in kaddish
	\item Reciting the first verse of shema\space with the congregation
	\item Reciting the blessing after using the toilet, on thunder, or on lightening
	\item Respond to a respected person's greeting, or greet a person out of fear
	\item Between paragraphs: greet a respected person, or respond to anyone's greeting
	\item Receive an aliya (though such a person shouldn't be given an aliya unless they're the only Kohen or Levi)
\end{itemize}

In all these cases it is preferable to interrupt between paragraphs if possible, and the interruption should be between verses.

Between Pesukei DeZimra and Sha\heth arit it is permitted to make communal announcements that cannot wait until after Full Kaddish \parencite*[16:2]{PH}.  Otherwise, any interruption at this point is forbidden, besides those permitted during Pesukei DeZimra.

\subsection{During the Shema and its Blessings}

During the Shema and its blessings, interruptions are forbidden with the following exceptions (based on \cite*[``Table of Permitted Responses"]{Koren} and \cite*[16:5]{PH}):
\begin{itemize}
	\item The congregational response to kedusha (but only the lines beginning \hebword{קדוש} and \hebword{ברוך}) and barekhu
	\item Reciting only the words \hebword{מודים אנחנו לך} from modim derabbanan
	\item Answering \hebword{אמן יהי שמה רבא וכו׳} in kaddish
	\item Receive an aliya (but not in the middle of the first verse of the Shema)
	\item Respond to a respected person's greeting, or greet a person out of fear
	\item Between paragraphs, greet a respected person, or respond to anyone's greeting
	\item Between paragraphs, recite the blessing on thunder and lighting
	\item Between paragraphs, answer \hebword{אמן} to a blessing
\end{itemize}

\subsection{During the \amidah}

Interruptions during the \amidah are forbidden, even congregational responses to Kaddish and Kedusha. It is considered an interruption to put on a tallit if it has fallen off (though it may be adjusted, OC 97:4).  It is permitted to walk to somewhere else if there is a distraction preventing prayer in the original location \parencite*[17:15]{PH}.

If an interruption is long enough that one could have finished the \amidah in the length of the interruption, the \amidah must be started again (PH Prayer 18:1).

\section{Time-Related Errors in Liturgy}

In all these cases, if the shalia\heth\space tzibbur makes a mistake in their silent \amidah they need not repeat it even if it would normally be required if this would cause undo delay for the community \parencite*[19:13]{Kitzur}

\subsection{Requests for Rain}

\paragraph{\textit{Mashiv HaRua\heth}}

Mashiv haRua\heth\space is recited beginning at Musaf on the first day of Pesa\heth\space until Musaf on Shemini Atzeret.  If it is omitted in winter or recited in summer, one must begin the \amidah again (unless the berakha has not been completed, in which case resume from the point of the error).  But if ``Morid HaTal" was recited in summer, the \amidah may be continued as normal \parencite*[18:4-5 citing OC 114]{PH}.

If Mashiv haRua\heth was recited during \arvit or sha\heth arit on Shemini Atzeret, or omitted on Pesa\heth , the \amidah need not be repeated \parencite*[19:2,4]{Kitzur}.

\paragraph{\textit{Tal uMatar}}

The request for rain in the weekday \amidah is recited during the rainy season.  In the Land of Israel it is begun on the 7th of Mar\heth eshvan at \arvit\space (i.e. the beginning of the 7th) \parencite*[18:5 citing OC 117]{PH}.

Elsewhere it is begun 60 days after the Autumnal Equinox, which for this purpose is reckoned according to the Julian calendar (Tekufat Shemuel).  In the 21st century this works out to 'Arvit on December 4th, or December 5th in years immediately before a civil leap year.  It moves one day on the Gregorian calendar later every time there is a Julian leap year that is not a Gregorian leap year, which occurs every Gregorian year that is divisible by 100 but not by 400. When such a year occurs, treat the year before as the year before a leap year.  Then in the non-leap year the request for rain moves one year later.  For example, in the year 2099 the request will begin December 5th, and on December 5th (or 6th, in years before a leap year) thereafter until 2198.

In both locations it is said until Passover. If it was forgotten and the blessing was concluded, it should be inserted in the blessing of \hebword{שמע קולינו} (before \hebword{עננו} if on a fast). If that blessing was concluded, return to the blessing of ``mevarekh hashanim".  If the \amidah was completed, it must be recited again \parencite*{PH}.

\subsection{Shabbat and Festival Liturgical Texts}

On Shabbat and Festivals, the blessing of Hashkiveinu is concluded with \hebword{הפורס סכת שלום וכו׳} instead of the weekday text, \hebword{שומר עמו ישרﭏ לעד}.  If the weekday text was said by mistake and the error was not caught immediately, one need not repeat the blessing.

If the weekday Amidah was mistakenly said on Shabbat or a Festival and the mistake was realized during the middle portion of the Amidah, the blessing currently being said is completed, and the middle blessing for Shabbat or Festivals should be recited and the Amidah continued as usual \parencite*[76:16]{Kitzur}.  This does not apply to Musaf, where one should not complete the blessing currently being recited \parencite*[76:19]{Kitzur}.  If the mistake was realized during the concluding three blessings, stop and return to the middle blessing for Shabbat or Festivals and continue as usual. If the Amidah was already completed, recite the Amidah from the beginning with the correct text \parencite*[76:18]{Kitzur}.

If the Shabbat Amidah was recited on a Festival or the reverse, if the central blessing was recited incorrectly one must return to the beginning of the central blessing of the Amidah.  If the Amidah was completed it must be recited over again.  The same applies to erroneously including or omitting reference to Shabbat in the concluding berakha, and to mentioning the wrong holiday \parencite*[2:3:4]{PHYT}

If the text of the Amidah was substituted on Shabbat between Arvit, Sha\d{h}arit, and Min\d{h}a, and the mistake was not realized until after concluding the blessing \hebword{מקדש השבת}, the Amidah need not be repeated.  This does not apply to Musaf, where reciting Musaf instead of another prayer, or another prayer in place of Musaf, requires reciting the correct Amidah \parencite*[76:21-22]{Kitzur}.

\subsection{Holiday Liturgical Additions}

Havdalah in the \amidah is recited at the end of Shabbat and Festivals.  If forgotten the \amidah is not repeated, since Havdalah will also be said separately \parencite*[18:2 citing OC 422]{PH}.

Ya'aleh veYavo is recited on festivals and Rosh \bigheth odesh.  If omitted, the \amidah must be repeated, except at \arvit\space on Rosh \bigheth odesh.  If the omission is realized before the \amidah is complete, return to Retzei and continue from there \parencite{PH}.

Al Hanisim is recited on \bigheth anukkah and Purim.  If omitted the \amidah is not repeated \parencite*[citing OC 682]{PH}.

If Aneinu, recited on fast days, is forgotten, the \amidah is not repeated \parencite*[citing OC 565]{PH}. It may be inserted in \hebword{שמע קולינו}, or if it is forgotten there, during the paragraph following the \amidah before taking three steps backwards.

During the Ten Days of Repentance, if \hebword{הﭏ הקדוש} is forgotten and the error is not fixed within a moment, return to the beginning of the \amidahnospace.  Other additions to not require repeating the \amidahnospace.

If a holiday addition was mistakenly recited on a weekday, return to the beginning of the berakha if the berakha has not been concluded.  If the berakha has been completed, continue as usual without returning or repeating.

\subsection{Cases of Doubt}

If one is unsure whether they included a special insertion to the \amidah that is required, they must assume they omitted it and repeat the \amidahnospace.  The same is true for the requests for rain (or lack thereof) within 30 days of beginning / ending including the request.  In such a case, one should have in mind that if they did recite the correct text, the repeated text should be considered a voluntary extra prayer \parencite*[18:6]{PH}.

\section{Birkat Hamazon}

\subsection{When to Recite Birkat Hamazon}

Birkat Hamazon is recited after a meal where an olive's size of bread is eaten. It should be recited within 72 minutes of the end of the meal.  If this time already elapsed, it may be recited as long as one is not yet hungry \parencite*[44:8]{Kitzur}.

Birkat Hamazon should be recited at the same location as the meal. If a person forgot and went somewhere else, they may eat some bread at the new location and recite Birkat Hamazon there. If not, they must return to the location of the meal if possible.  If that is impossible or the time to recite Birkat Hamazon would pass, it is recited at the new location \parencite*[44:9]{Kitzur}.

\subsection{Errors in Additions to Birkat Hamazon}

In all below cases, if a berakha has been partially said, if only \hebword{ברוך אתה ה׳} has been said, conclude the line \hebword{לַמְּדֵֽנִי חֻקֶּֽיךָ}, which forms a complete verse (Psalms 119:12), and proceed as though the berakha had not been begun.  If \hebword{ﭏהינו} was said, the berakha cannot be un-said, and either continue with the berakha or return to the beginning, depending on the rule for the specific case.

\paragraph{\textit{Retzei and Ya'aleh veYavo}} Retzei is added on Shabbat, and Ya'aleh veYavo is added on Festivals and Rosh \d{H}odesh.  If forgotten and the berakha of \hebword{בונה ירושלים} has not been said, return and say Retzei or Ya'aleh veYavo and continue as usual from there.  If forgotten and the next berakha has already been begun (the one that begins \hebword{ברוך אתה ה׳ ﭏהינו מלך העולם הﭏ אבינו וכו׳}): \begin{itemize}
	\item On Shabbat and festivals at the evening or first daytime meal, repeat Birkat Hamazon
	\item On Shabbat and festivals at the third meal, do not repeat Birkat Hamazon
	\item On Rosh \d{H}odesh, do not repeat Birkat Hamazon
	\item On \d{H}ol HaMo'ed, do not repeat Birkat Hamazon
\end{itemize}

If the berakha of \hebword{בונה ירושלים} has been completed but the next berakha has not begun, add a compensatory blessing.  The texts are as follows:

\hebword{בָּרוּךְ אַתָּה יְיָ ﭏֱֹהֵֽינוּ מֶֽלֶךְ הָעוֹלָם אֲשֶׁר נָתַן...}

\begin{itemize}
	\item If Retzei was forgotten on Shabbat:\\\hebword{שַׁבָּתוֹת לִמְנוּחָה לְעַמּוֹ יִשְׂרָאֵל בְּאַהֲבָה לְאוֹת וְלִבְרִית׃ בָּרוּךְ אַתָּה יְיָ מְקַדֵּשׁ הַשַּׁבָּת׃}
	\item If Ya'aleh veYavo was forgotten on Festivals. On Shabbat if Retzei was forgotten as well, add the words in parentheses.\\\hebword{
		(שַׁבָּתוֹת לִמְנוּחָה לְעַמּוֹ יִשְׂרָאֵל בְּאַהֲבָה לְאוֹת וְלִבְרִית וֽ)יָמִים טוֹבִים לְשָׂשׂוֹן וּלְשִׂמְחָה אֶת־יוֹם חַג ... הַזֶּה׃ בָּרוּךְ אַתָּה יְיָ מְקַדֵּשׁ (הַשַּׁבָּת וְ)יִשְׂרָאֵל וְהַזְּמַנִּים׃}
	\item If Ya'aleh veYavo was forgotten on Rosh \d{H}odesh.  On Shabbat if Retzei was forgotten as well, add the words in parentheses. \\\hebword{
		(שַׁבָּתוֹת לִמְנוּחָה לְעַמּוֹ יִשְׂרָאֵל בְּאַהֲבָה לְאוֹת וְלִבְרִית וֽ)רָאשֵׁי חֳדָשִׁים לְזִכָּרוֹן׃ (בָּרוּךְ אַתָּה יְיָ מְקַדֵּשׁ הַשַּׁבָּת וְיִשְׂרָאֵל וְרָאשֵׁי חֳדָשִׁים׃)}
\end{itemize}

\paragraph{\textit{Al haNisim}}

Al haNisim is recited on \d{H}anukka and Purim.  If forgotten, Birkat haMazon is not repeated.  If the mistake is realized before completing Birkat Hamazon, add to the \hebword{הרחמן} section \hebword{הָרַחֲמָן הוּא יַעֲשֶׂה לָנוּ נִסִּים וְנִפְלָאוֹת כְּמוֹ שֶׁעָשָׂה לַאֲבוֹתֵינוּ בַּיָּמִים הַהֵם בַּזְּמַן הַזֶּה} followed by the text for that day, beginning with either \hebword{בימי מרדכי} or \hebword{בימי מתתיהו} \parencite*[44:16]{Kitzur}.

%\section{Priority List for Aliyot}
%
%A Kohen recieves the first aliya, and a Levi the second.  If there is no Kohen, a Levi need not be called first (or second).  If there is no Levi, the same Kohen is called for the first two aliyot.  In a case of great need the Kohen may be asked to leave so a Yisrael can be called instead.
%
%There are variations in local custom of the priority list for aliyot.  This is one such list\parencite*[78:11]{Kitzur}:\begin{enumerate}
%	\item A person getting married that day
%	\item A 
%	\item A child reaching Bar-Mitzva, either on a weekday or on the following Shabbat
%	\item A new parent
%	\item On Shabbat, a person who got married the previous week on Wednesday or later
%	\item Someone observing a Yortzeit
%	\item On Shabbat, a parent whose son will get a Brit Mila in the following week
%	\item Someone who needs to recite Gomel or is leaving on a journey
%\end{enumerate}
%
%A person may waive their precedence to recieve an aliya.
%
%\section{Priority List for the Amud}
%
%On weekdays that are not festive days, it is customary to give mourners priority to lead services.  In former times this was also the priority list to recite kaddish, but nowadays in most commnuities kaddish is recited by mourners collectively.\begin{enumerate}
%	\item 
%\end{enumerate}

\input{sefer_torah_section}

\printbibliography

\end{document}